\textbf{微机结构}
\begin{itemize}
	\item Input 输入
	\item Output 输出
	\item Memory 存储器
	\item ALU 算术逻辑单元
	\item Control unit 控制单元
\end{itemize}

\begin{table*}
	\centering
	\caption{微机概念差异}
\begin{tabular}{|>{\bfseries}l|p{8cm}|}
	\hline
	微处理器 Microprocessor & 可以被微缩成集成电路规模的CPU电路,包含ALU,CU,寄存器 \\
	\hline
	微型计算机 Mircrocomputer & 微处理器,存储器,I/O,总线 \\
	\hline
	微型计算机系统 Microcomputer system & 以微型计算机为主体,配上I/O及系统软件就构成了微型计算机系统。 \\
	\hline
	微控制器 Microcontrollers & A microcontroller has a CPU in addition to a fixed amount of RAM, ROM, I/O ports on one single chip (\emph{e.g.} Cortex) \\
	\hline
	嵌入式系统 Embedded Systems & An embedded system uses a microcontroller or a microprocessor to do one task and one task only\\
	\hline
\end{tabular}
\end{table*}

\textbf{指令集}:CISC(1-n个字),RISC(1个字)

\textbf{字}:CPU一次可以处理的最大比特数

\textbf{位扩展}:同一地址的位扩展,满足一个字的输出

\textbf{字扩展}:增大字的量,选择不同的字,满足存储量需求

\begin{table*}
	\begin{minipage}{0.3\textwidth}
		\centering
		\caption{总线类型}
		\begin{tabular}{|c|c|c|}
			\hline
			\bfseries 类型 &\bfseries 仲裁 &\bfseries 时序 \\
			\hline
			单工 & 集中式 & 同步 \\
			多工 & 分布式 & 异步 \\
			\hline
		\end{tabular}
	\end{minipage}
	\begin{minipage}{0.7\textwidth}
		\centering
		\caption{总线结构}
		\begin{tabular}{|>{\bfseries}c|l|l|}
			\hline
			& \bfseries 优点 &\bfseries 缺点 \\
			\hline
			单线结构 & 简单 & 吞吐量低 \\
			\hline
			CPU-Central 双线结构 & 数据传输率高 & I/O与内存需要经过CPU \\
			\hline
			Memory-Central 双线结构 & CPU性能好 吞吐量高 &  \\
			\hline
		\end{tabular}
	\end{minipage}
\end{table*}

